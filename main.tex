\documentclass[phd,tocprelim]{cornell}

%
% tocprelim option must be included to put the roman numeral pages in the
% table of contents
%
% The cornellheadings option will make headings completely consistent with
% guidelines.
%
% This sample document was originally provided by Blake Jacquot, and
% fixed up by Andrew Myers.
%
%Some possible packages to include
\usepackage{graphicx,pstricks}
\usepackage{graphics}
\usepackage{moreverb}
\usepackage{subfigure}
\usepackage{epsfig}
\usepackage{subfigure}
\usepackage{hangcaption}
\usepackage{txfonts}
\usepackage{palatino}
% \usepackage{ctex}

%if you're having problems with overfull boxes, you may need to increase
%the tolerance to 9999
\tolerance=9999

\bibliographystyle{plain}
%\bibliographystyle{IEEEbib}

\renewcommand{\caption}[1]{\singlespacing\hangcaption{#1}\normalspacing}
\renewcommand{\topfraction}{0.85}
\renewcommand{\textfraction}{0.1}
\renewcommand{\floatpagefraction}{0.75}

\title {Fast Semistochastic Heat-Bath Configuration Interaction}
\author {Junhao Li}
\conferraldate {June}{2019}
\degreefield {Ph.D.}
\copyrightholder{Junhao Li}
\copyrightyear{2019}

\begin{document}

\maketitle
\makecopyright

\begin{abstract}
In this thesis, I present my work on the fast semistochastic heat-bath configuration interaction, which is an efficient algorithm for solving the many-body solving electronic structure calculations and computational research on large clusters. Arrow provides two sets of features: a quantum solver based on the Fast Semistochastic Heatbath Configuration Interaction (Fast SHCI) algorithm recently developed in our group, and several generic distributed computing building blocks.

Fast SHCI [Junhao Li et al., JCP 2018] is more than 100 times faster than other algorithms in its category (selected CI plus perturbation theory), and also much faster than other essentially exact algorithms for most chemical systems. This thesis provides an in-depth description of the Fast SHCI algorithm and its implementation in Arrow. I use Arrow to compute electronic structures of several chemical systems and the homogeneous electron gas.  Some of these calculations are more accurate than those achieved by other high-order quantum chemistry methods.  Others treat systems larger than those that can be treated by other equally accurate methods.

Arrow uses a modular design, which not only makes the library highly extensible but also contributes several generic distributed computing building blocks to the open-source community. The high-level components that come with Arrow include:
(1) A data serialization library that is upto four times faster and produces more compact serialized messages than widely used libraries in industry;
(2) A highly optimized cluster computing library that is upto eight times faster than Apache Spark on common data processing tasks;
(3) A modified version of Google Test for testing distributed computing modules.
In this thesis, I discuss my design and implementation of all of them.

Finally, I also provide a brief discussion of the usability of general software engineering best practices for the development of medium-scale scientific software packages with lessons learned from designing, developing, and leading the development of Arrow. Medium-scale scientific software packages are common in scientific research where a small group of researchers works on the same code base. Due to the differences in the requirements, some best practices that are common in industry need to be adjusted to be useful for these projects.

Your abstract goes here. Make sure it sits inside the brackets. If not,
your biosketch page may not be roman numeral iii, as required by the
graduate school.
\end{abstract}

\begin{biosketch}
Junhao Li was born and grew up in Shanghai, China.
From a young age, he had a strong interest in science and engineering, and enjoyed disassembling all kinds of home appliances.

Junhao attended Shanghai Jiao Tong University from 2009, graduating with a B.S. in physics and a B.S.E in computer science in 2013.
While an undergraduate, he did research in various fields, including semiconductor fabrication, photovoltaics, finite element analysis of electromagnetic field, density functional theory, and social networks.

In the fall of 2013, Junhao came to Cornell University to persue a Ph.D. in physics.
During his Ph.D, he mainly worked with professor Cyrus Umrigar on the development and application of highly accurate quantum chemistry methods.
He developed the fast heat-bath configuration interaction method, which is faster than other methods in its category by more than an order of magnitude, and achieved significantly higher accuracy than other well developed methods, such as DMRG and FCIQMC, on several systems.
he also did research in molecular dynamics and defects with professor James Sethna.

After obtaining a master's degree at Cornell, Junhao spent a summer as an intern at Google in California in 2016.
He then returned to Ithaca, brought back the engineering merits he learned from Google, and developed a highly efficient and extensible quantum chemistry package, Arrow, as well as several open-sourced generic high performance computing libraries.
\end{biosketch}

\begin{dedication}
This document is dedicated to all Cornell graduate students.
\end{dedication}

\begin{acknowledgements}
First of all, I would like to thank my advisor Professor Cyrus Umrigar.
His commitment to physics research is exemplary and inspiring.
He led me into the field of quantum chemistry, which is an indispensable foundation of highly accurate numerical simulations.
All my knowledge of quantum Monte Carlo are inherited from him, and my work on heat-bath configuration interaction is impossible without his guidance and support.
He also taught me valuable research skills, and helped me go through the process of publishing my first first-author paper, on a top academic journal.
I am also extremely grateful to him for always being supportive in the decisions I made, the ideas I wanted to try, and being tolerant and direct with the mistakes I made.
He made my six years of Ph.D life a truly rewarding experience.

I would also like to thank Professor James Sethna.
It was a pleasure to work with Jim on defects and molecular dynamics.
His enthusiasm and optimism inspires me whenever I face chanllenging problems.
I also learned a lot from his tremendous physics and data insights.

I also thank all the graduate students I worked with, especially Adam Holms, Matt Otten, and Matt Bierbaum.
They helped me getting started in my research, patiently answered all my questions, and gave me valuable advices on my projects.
I could not imagine how much longer it would take me to get start and how many detours I might have taken without their help.

Special thanks to my parents for giving me birth and raising me up.
They gave me a warm family as I grew up, and consistently support as I persue my degree at Cornell.
Thanks also to my elementary school, middle school, high school, and undergraduate teachers and friends back in China for all the invaluable lessons I learned from them, the wonderful memories I had with them, and helps and encouragements I received from them.

Finally, I acknowledge the financial support from the Cornell physics department, National Science Foundation and the Air Force Office of Scientific Research, and the computing resources support from Pittsburg Computing Center, Argonne National Lab, NERSC, and Google Cloud.
Thank you and I hope you are proud.
\end{acknowledgements}

\contentspage
\tablelistpage
\figurelistpage

\normalspacing \setcounter{page}{1} \pagenumbering{arabic}
\pagestyle{cornell} \addtolength{\parskip}{0.5\baselineskip}

\chapter{Introduction}

\section{SECTION 1}
The text for Section 1 goes here, without brackets.

\section{SECTION 2}
Section 2 text.

\subsection{Subsection heading goes here}

Subsection 1 text

\subsubsection{Subsubsection 1 heading goes here}
Subsubsection 1 text

\subsubsection{Subsubsection 2 heading goes here}
Subsubsection 2 text

\section{SECTION 3}
Section 3 text. The dielectric constant at the air-metal interface
determines the resonance shift as absorption or capture occurs.

\begin{equation}
k_1=\frac{\omega }{c({1/\varepsilon_m + 1/\varepsilon_i})^{1/2}}=k_2=\frac{\omega
sin(\theta)\varepsilon_{air}^{1/2}}{c}
\end{equation}

\noindent
where $\omega$ is the frequency of the plasmon, $c$ is the speed of
light, $\varepsilon_m$ is the dielectric constant of the metal,
$\varepsilon_i$ is the dielectric constant of neighboring insulator,
and $\varepsilon_{air}$ is the dielectric constant of air.

\chapter{Alice in Wonderland}

\section{The Black Kitten}
  One thing was certain, that the WHITE kitten had had nothing to
do with it:---it was the black kitten's fault entirely~\cite{aiw}.  For the
white kitten had been having its face washed by the old cat for
the last quarter of an hour (and bearing it pretty well,
considering); so you see that it COULDN'T have had any hand in
the mischief.

  The way Dinah washed her children's faces was this:  first she
held the poor thing down by its ear with one paw, and then with
the other paw she rubbed its face all over, the wrong way,
beginning at the nose:  and just now, as I said, she was hard at
work on the white kitten, which was lying quite still and trying
to purr---no doubt feeling that it was all meant for its good.

  But the black kitten had been finished with earlier in the
afternoon, and so, while Alice was sitting curled up in a corner
of the great arm-chair, half talking to herself and half asleep,
the kitten had been having a grand game of romps with the ball of
worsted Alice had been trying to wind up, and had been rolling it
up and down till it had all come undone again; and there it was,
spread over the hearth-rug, all knots and tangles, with the
kitten running after its own tail in the middle.

\section{The Reproach}

  `Oh, you wicked little thing!' cried Alice, catching up the
kitten, and giving it a little kiss to make it understand that it
was in disgrace.  `Really, Dinah ought to have taught you better
manners!  You OUGHT, Dinah, you know you ought!' she added,
looking reproachfully at the old cat, and speaking in as cross a
voice as she could manage---and then she scrambled back into the
arm-chair, taking the kitten and the worsted with her, and began
winding up the ball again.  But she didn't get on very fast, as
she was talking all the time, sometimes to the kitten, and
sometimes to herself.  Kitty sat very demurely on her knee,
pretending to watch the progress of the winding, and now and then
putting out one paw and gently touching the ball, as if it would
be glad to help, if it might.

  `Do you know what to-morrow is, Kitty?' Alice began.  `You'd
have guessed if you'd been up in the window with me---only Dinah
was making you tidy, so you couldn't.  I was watching the boys
getting in stick for the bonfire---and it wants plenty of
sticks, Kitty!  Only it got so cold, and it snowed so, they had
to leave off.  Never mind, Kitty, we'll go and see the bonfire
to-morrow.'  Here Alice wound two or three turns of the worsted
round the kitten's neck, just to see how it would look:  this led
to a scramble, in which the ball rolled down upon the floor, and
yards and yards of it got unwound again.

  `Do you know, I was so angry, Kitty,' Alice went on as soon as
they were comfortably settled again, `when I saw all the mischief
you had been doing, I was very nearly opening the window, and
putting you out into the snow!  And you'd have deserved it, you
little mischievous darling!  What have you got to say for
yourself?  Now don't interrupt me!' she went on, holding up one
finger.  `I'm going to tell you all your faults.  Number one:
you squeaked twice while Dinah was washing your face this
morning.  Now you can't deny it, Kitty:  I heard you!  What that
you say?' (pretending that the kitten was speaking.)  `Her paw
went into your eye?  Well, that's YOUR fault, for keeping your
eyes open---if you'd shut them tight up, it wouldn't have
happened.  Now don't make any more excuses, but listen!  Number
two:  you pulled Snowdrop away by the tail just as I had put down
the saucer of milk before her!  What, you were thirsty, were you?

\chapter{Chapter 3}

\chapter{Chapter 4}

\appendix
\chapter{Chapter 1 of appendix}
Appendix chapter 1 text goes here

\bibliography{sampleThesis}

\end{document}
