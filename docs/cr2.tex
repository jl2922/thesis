\chapter{Chromium Dimer Simulation with Fast SHCI}

\section{Introduction}

\section{Equilibrium Geometry}

\label{results}

In this section, we apply SHCI to the chromium dimer, which is a challenging strongly-correlated system
that has been studied using a variety of methods~\cite{Scu-JCP-91,KurYan-JCP-11,PurZhaKra-JCP-15,MaManOlsGag-JCTC-16,VanMalVer-JCTC-16,GuoWatHuSunCha-JCTC-16}.
We will publish the potential energy surface in a separate publication; here instead our goal is just
to use it as a test case for the improved SHCI method.
%We use a cc-pVDZ basis with active semi-core electrons.
We use a relativistic exact two-component (X2C) Hamiltonian, the cc-pVDZ-DK basis, and we correlate the valence and the semi-core electrons.
This gives an active space of (28e, 76o) and a Hilbert space of $5\times10^{29}$ determinants, which is far beyond the reach of FCI.
We show how we obtain an accurate estimate of the FCI energy in this large active space with our improved SHCI algorithm.

We use PySCF~\cite{SunCha_etal_PySCF-ComMolSci-18} to generate the molecular orbital integrals for orbitals that minimize the HCI variational
energy for $\epsilon_1=2\times 10^{-4}$~Ha, using the method of Ref.~\cite{SmiMusHolSha-JCTC-17}.
We perform SHCI with several $\epsilon_1$ values from $5\times10^{-5}$ to $3\times10^{-6}$~Ha.
The Hamiltonian matrix is constructed only once.
% The Hamiltonian matrix is stored in memory and so needs to be constructed only once.
%We keep the ratio between $\epsilon_2$ and $\epsilon_1$ to $10^{-6}$ and the target error for the stochastic perturbation to 0.01 mHa.
We use very small values of $\epsilon_2 = 10^{-6} \epsilon_1$  to ensure that the perturbative correction is exceedingly well converged,
and choose the target error for the stochastic perturbation energy to be $10^{-5}$~Ha.

%The improved SHCI is fast enough that we can use over two billion variational determinants and include at least trillions of effective perturbative determinants.
The improved SHCI is fast enough that we can use over two billion variational determinants, and stochastically include the contributions of at least trillions of perturbative determinants.
The largest variational calculation, where we iteratively find and diagonalize 2 billion determinants
for $\epsilon_1=3.0\times10^{-6}$~Ha, takes only one day on 8 nodes, each of which has 4 Intel Xeon E7-8870 v4 CPUs.
The corresponding perturbative calculation takes only 6 hours using only one of these nodes. 
During that perturbative calculation, we skip the deterministic step, perform a pseudo-stochastic step with $\epsilon_2^{\rm psto}=1\times10^{-7}$~Ha, and a stochastic step with $\epsilon_2=3\times10^{-12}$~Ha.
{\color{black}
We skip the deterministic step here because $\epsilon_1=3 \times 10^{-6}$ is already close to our
default $\epsilon_2^{\rm dtm}$ of $2 \times 10^{-6}$ so skipping this won't affect the efficiency
of subsequent steps much.
}
The pseudo-stochastic step uses 25 batches, each of which has about 8.9 billion determinants.
We evaluate only one of them, from which we obtain an estimate of the total correction for all the 25 batches (223 billion determinants) to be -0.011681(1)~Ha.
Since the estimated error is already much smaller than our target error, we skip the remaining 24 batches.
The pseudo-stochastic step takes 1.6 hours.
The stochastic step uses 128 batches and 6 million variational determinants in each sample,
which results in about 3.7 billion determinants per batch.
We use 10 samples and obtain the additional correction from $\epsilon_2=3.0\times10^{-12}$~Ha to be -0.001203(6)~Ha.
The combined uncertainty of the entire semistochastic perturbation stage is 6~$\mu$Ha.
It is hard to estimate how many determinants are stochastically included for $\epsilon_2=3\times10^{-12}$~Ha,
so we estimate a lower bound with $\epsilon_2^{\rm psto}=1.4\times10^{-8}$~Ha and obtain 1.8 trillion unique perturbative determinants.
Hence, with $\epsilon_2=3\times10^{-12}$~Ha (the value we are actually using) we stochastically estimate contributions from
at least trillions of unique perturbative determinants and obtain better than $10^{-5}$~Ha statistical uncertainty in 6 hours using only one node.
%why?
%This brings the accuracy selected-CI to a whole new level and enables us to obtain an estimation of the FCI energy with sub-millihartree uncertainty in this large active space.

These large calculations enable us to obtain an estimate of the FCI energy with sub-millihartree uncertainty in this large active space.
Table~\ref{tab:Cr2} reports the results.


\begin{table}[h]
  \begin{tabular}{| c | c | c | c |}
  \hline
  $\epsilon_{1}$ (Ha) & $N_\V$ & $E_{\rm var}$ (Ha) & $E_{\rm total}$ (Ha) \\
  \hline\hline
  $5.0\times10^{-5}$ & 24M & -2099.863816 & -2099.909741(7) \\
  \hline
  $3.0\times10^{-5}$ & 53M & -2099.875327 & -2099.912356(7) \\
  \hline
  $2.0\times10^{-5}$ & 102M & -2099.883027 & -2099.914132(8) \\
  \hline
  $1.0\times10^{-5}$ & 309M & -2099.893761 & -2099.916595(1) \\
  \hline
  $7.0\times10^{-6}$ & 539M & -2099.898165 & -2099.917540(1) \\
  \hline
  $5.0\times10^{-6}$ & 911M & -2099.901781 & -2099.918306(3) \\
  \hline
  $3.0\times10^{-6}$ & 2.00B & -2099.906322 & -2099.919205(6) \\
  \hline
  0.0 (Extrap.) & - & \multicolumn{2}{ c |}{-2099.9224(6)} \\
  \hline
  \end{tabular}
%   \begin{tabular}{| c | c | c | c | c |}
%   \hline
%   Basis & $\epsilon_{1}$ & $|V|$ & $E_{\rm var}$ (Ha) & $E_{\rm total}$ (Ha) \\
%   \hline\hline
%   \multirow{8}{*}{TZ} & $5.0\times10^{-4}$ & 3.4M & -2100.000182 & -2100.280436(4) \\
%    \cline{2-5}
%    & $3.0\times10^{-4}$ & 6.9M & -2100.050564 & -2100.292239(10) \\
%    \cline{2-5}
%    & $2.0\times10^{-4}$ & 13.0M & -2100.086849 & -2100.301229(4) \\
%    \cline{2-5}
%    & $1.0\times10^{-4}$ & 40.7M & -2100.140331 & -2100.314712(6) \\
%    \cline{2-5}
%    & $5.0\times10^{-5}$ & 132M & -2100.184546 & -2100.326457(4) \\
%    \cline{2-5}
%    & $4.0\times10^{-5}$ & 192M & -2100.197051 & -2100.329894(4) \\
%    \cline{2-5}
%    & $3.0\times10^{-5}$ & 316M & -2100.212485 & -2099.889577(9) \\
%    \cline{2-5}
%    & $2.0\times10^{-5}$ & 644M & -2100.232505 & -2099.889577(9) \\
%    \cline{2-5}
%    & $1.5\times10^{-5}$ & 1.06B & -2100.245442 & -2099.889577(9) \\
%    \cline{2-5}
%    & 0 (Ext.) & $6\times10^{42}$ & \multicolumn{2}{ c |}{-2099.889577(9)} \\
%    \hline
%   \multirow{8}{*}{QZ} & $5.0\times10^{-4}$ & 2.6M & -2100.093991 & -2100.349945(19) \\
%    \cline{2-5}
%    & $3.0\times10^{-4}$ & 5.8M & -2100.136962 & -2100.356925(10) \\
%    \cline{2-5}
%    & $2.0\times10^{-4}$ & 11.6M & -2100.168104 & -2100.364967(8) \\
%    \cline{2-5}
%    & $1.0\times10^{-4}$ & 38.3M & -2100.214213 & -2100.378015(6) \\
%    \cline{2-5}
%    & $5.0\times10^{-5}$ & 122M & -2100.251234 & -2100.388998(7) \\
%    \cline{2-5}
%    & $4.0\times10^{-5}$ & 177M & -2100.261739 & -2100.392201(5) \\
%    \cline{2-5}
%    & $3.0\times10^{-5}$ & 293M & -2100.275010 & - \\
%    \cline{2-5}
%    & $2.0\times10^{-5}$ & 603M & -2100.292767 & - \\
%    \cline{2-5}
%    & $1.5\times10^{-5}$ & 1.01B & -2100.304661 & -2100.404533(7) \\
%    \cline{2-5}
%    & 0 (Ext.) & $5\times10^{46}$ & \multicolumn{2}{ c |}{-2099.889577(9)} \\
%    \hline
%   \end{tabular}
  \caption{Results for Cr$_2$ at r=1.68\AA\ in the cc-pVDZ-DK basis.
  The active space is (28e, 76o).
  $N_\V$ is the number of variational determinants.
  $\epsilon_2 = 10^{-6} \epsilon_1$.
  We use weighted quadratic extrapolation, shown in Fig.~\ref{fig:extrapolation}, to obtain the FCI limit
  corresponding to $\Delta E=0$.}
  \label{tab:Cr2}
\end{table}

\begin{figure}
  \includegraphics[width=\linewidth]{figs/extrapolate.eps}
  \caption{Weighted quadratic extrapolation of the Cr$_2$ ground state energy.
  The weight of each point is $(E_{\rm var} - E_{\rm tot})^{-2}$.
  The extrapolated energy is $-2099.9224(6)$, where the uncertainty comes from the difference between linear extrapolation and quadratic extrapolation.
  The p-DMRG extrapolation and the CCSD(T) value are also shown.
}
  \label{fig:extrapolation}
\end{figure}

We extrapolate our results using a weighted quadratic fit
and obtain for the ground state energy, $-2099.9224$~Ha as $\Delta E\to0$.
The weight of each point is $(E_{\rm var} - E_{\rm tot})^{-2}$.
Fig.~\ref{fig:extrapolation} shows the computed energies and the extrapolation.
We also perform a weighted linear fit and use the difference of the extrapolated values from the quadratic and the linear fits (0.6~mHa) as the uncertainty.
%If we look at the trends of the data points in Fig.~\ref{fig:extrapolation}, we can see that 0.55~mHa is a reasonable estimation of the uncertainty.
%In summary, the estimated FCI energy of Cr$_2$ in the cc-pVDZ basis with active semi-core electrons is $-2099.9224(6)$~Ha.
In summary, the estimated FCI energy of Cr$_2$ in the cc-pVDZ-DK basis with 28 correlated electrons and the relativistic X2C Hamiltonian
%(the rest of the electrons are frozen in
%Hartree-Fock orbitals)
is $-2099.9224(6)$~Ha.

\begin{figure}
  \includegraphics[width=\linewidth]{figs/excitation.eps}
  \caption{Contribution from each excitation level to the variational wavefunction for Cr$_2$ with $2 \times 10^9$ determinants.
  Determinants with up to 15 excitations are present in the variational wavefunction.
%  {\color{red} Leave out the top plot, since sums of squared magnitudes is meaningful, but number of nonzero magnitudes is a meaningless quantity? Well, it does mean that there were several higher-order excitations that are more important than lower-order ones.}
}
  \label{fig:excit}
\end{figure}

We compare our result with DMRG and p-DMRG, which are the only essentially exact methods that have been applied to this large active space of the chromium dimer.
The DMRG calculations use up to bond dimension $M=16000$ and obtain an extrapolated energy of $-2099.9195(27)$~Ha (default schedule) and $-2099.9192(24)$ (reverse schedule)~\cite{GuoLiCha-JCTC-18}.
%These two values are similar to our most accurate data point but higher than our extrapolated result by 3~mH, which is about one standard error of the DMRG results.
These two values are similar to our most accurate data point but higher than our extrapolated result by 3~mH, which is about the estimated error of the DMRG results.
%The p-DMRG calculations use up to $M=4000$ and obtain an extrapolated energy of $-2099.9201$~Ha~\cite{GuoLiCha-JCTC-18}.
The p-DMRG calculations use up to $M=4000$ and extrapolated energy obtained from a linear fit is $-2099.9201$~Ha~\cite{GuoLiCha-JCTC-18}.
If instead, we perform a weighted quadratic fit (shown in Fig.~\ref{fig:extrapolation}), the extrapolated energy is $-2099.9225$~Ha,
in perhaps fortuitously good agreement with our result of $-2099.9224(6)$~Ha.  However, the extrapolation uncertainty is larger than the SHCI extrapolation uncertainty.
In contrast, the CCSD(T) energy is considerably higher.
%In summary, our extrapolated value agrees with DMRG's and p-DMRG's extrapolated results but we achieve a smaller uncertainty.

%One of the merits of selected-CI is the ability to take higher-order excitations into account accurately.
%One of the merits of selected-CI methods is the ability to include higher-order excitations.
One of the merits of selected-CI methods is the ability to include all excitations, regardless of excitation level.
To see the contribution from each excitation level we plot the number of selected determinants and the $\sum_i \left|c_i\right|^2$ versus excitation level in Fig.~\ref{fig:excit}.
Determinants with excitation levels up to 15 excitations are present in the variational wavefunction even though we are using optimized orbitals.
(Using Hartree-Fock orbitals, we expect that determinants with even higher excitation levels will be present.)
%This implies that truncating the CI or the CC at the double or the triple excitation level may not give reliable results for such strongly correlated systems.
This implies that truncating the CI expansion at the double, triple or quadruple excitation levels (which is the most that is usually done in systematic
CI expansions), will give poor energies for such strongly correlated systems.

\section{Potential Energy Surface}


\section{Conclusion}